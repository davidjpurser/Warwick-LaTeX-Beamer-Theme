\documentclass[aspectratio=169]{beamer}

 
\usetheme[style=aubergine]{Warwick}
% \usetheme[]{Madrid}
\usepackage[UKenglish]{isodate}
\usepackage{listings} 


%Information to be included in the title page:
\title{An Example Presentation}
\author[Sirius Black]{Sirius Black\\ University of Warwick\\[0.5cm] Joint work with: Neville Longbottom, Hogwarts and Hermiony Granger, Ministry of Magic}
\institute[Warwick]{}
\date{1st May 2019}
 
 
\begin{document}
 


\begin{frame}
\maketitle

\centering
Muggle Orientation Evening

\end{frame}

 
\begin{frame}
\frametitle{Muggles}
We first explore what a Muggle is 

\begin{definition}[Muggle]
A person without Magic

\end{definition}


\begin{example}[Muggle]
The prime minister

\end{example}
\end{frame}

 
\begin{frame}
\frametitle{Wizards}
\begin{definition}[Wizards]
People with Magic
\end{definition} 

\pause
\begin{theorem}
No Muggle knows about the magical world
\end{theorem} 

\end{frame}

\begin{frame}[fragile]
\frametitle{Algorithm}
\framesubtitle{For the capture of Magical criminals}

\begin{lstlisting}[mathescape=true]
Call Aurors
while not(here):
	hide
\end{lstlisting}


\end{frame}

\begin{frame}
\frametitle{Rules of Magical People}

The following rules apply to all witches and wizards:
\begin{itemize}
\item Don't eat Muggles
\item Don't kill Muggles
\end{itemize}

\end{frame}

\begin{frame}
\frametitle{Summary}

\begin{block}{Results}
\begin{itemize}
	\item There are wizards
	\item There are Muggels
	\pause
	\item Don't get them confused.
\end{itemize}
\end{block}

\end{frame}
 
\end{document}

